\documentclass[10pt]{article}
\usepackage[a4paper, total={6in, 8in}]{geometry}
\usepackage{fullpage}
\usepackage{amsmath}
\usepackage{amssymb}
\usepackage[usenames,dvipsnames]{xcolor}
\usepackage[]{hyperref}
\usepackage[utf8]{inputenc}
% \usepackage[cjk]{kotex}
\usepackage{fontspec}
% \setmainfont{Helvetica}
\setmainfont{Noto Sans}
% \setmainfont{Noto Serif}

\usepackage[space]{xeCJK}
% \setCJKmainfont{Noto Sans CJK}
% \setCJKfamilyfont{kr}{Noto Sans CJK KR}
% \setCJKfamilyfont{jp}{Noto Sans CJK JP}
\setCJKsansfont{Noto Sans CJK KR}
\setCJKsansfont{Noto Sans CJK JP}
% \setCJKmainfont{Noto Serif CJK KR}
% \setCJKmonofont{Noto Sans Mono CJK KR}
% \usepackage{CJKutf8}

\usepackage[]{fancyhdr}
\hypersetup{
    colorlinks=true,
    citecolor=black,
    filecolor=black,
    linkcolor=NavyBlue,
    urlcolor=NavyBlue,
    breaklinks=true
}
\urlstyle{same}

\leftmargin=0.25in
\oddsidemargin=0.25in
\textwidth=6.0in
\topmargin=-0.25in
\textheight=9.25in

\raggedright

\pagenumbering{gobble} % suppresses page numbering. To stop suppression, change to 'arabic'.

\def\bull{\vrule height 0.8ex width .7ex depth -.1ex }


% DEFINITIONS FOR CV - defining some features to simplify template

\newenvironment{changemargin}[2]{
  \begin{list}{}{
    \setlength{\topsep}{0pt}
    \setlength{\leftmargin}{#1}
    \setlength{\rightmargin}{#2}
    \setlength{\listparindent}{\parindent}
    \setlength{\itemindent}{\parindent}
    \setlength{\parsep}{\parskip}
  }
  \item[]}{\end{list}
}

\newcommand{\lineover}{
	\begin{changemargin}{-0.05in}{-0.05in}
		\vspace*{-8pt}
		\hrulefill \\
		\vspace*{-2pt}
	\end{changemargin}
}

\newcommand{\header}[1]{
	\begin{changemargin}{-0.5in}{-0.5in}
		\scshape{#1}\\
  	\lineover
	\end{changemargin}
}

\newcommand{\contact}[5]{
	\begin{changemargin}{-0.5in}{-0.5in}
		\begin{center}
			{\Large \scshape {#1}}\\
      {#2} \\  {#3} \\ {#4} \\ {#5}
		\end{center}
	\end{changemargin}
}

\newcommand{\labdescription}[1]{
	\begin{changemargin}{0.15in}{0.15in}
    \smallskip
		{#1}
    \medskip
	\end{changemargin}
}

\newcommand{\labtitle}[3]{
	\textbf{#1}, \emph{#2} \hfill \emph{#3}\\
}

\newcommand{\presentation}[2]{
	{#1} \hfill \emph{#2}\\ \bigskip
}

\newcommand{\award}[2]{
	{#1} \hfill \emph{#2}\\ \medskip
}

\newcommand{\society}[2]{
	{#1} \hfill \emph{#2}\\ \medskip
}

\newcommand{\project}[2]{
	{#1} \hfill \emph{#2}\\ \medskip
}
\newcommand{\RED}[1]{\textcolor{red}{#1}}
\newcommand{\BLUE}[1]{\textcolor{blue}{#1}}

\newenvironment{body} {
	\vspace*{-16pt}
	\begin{changemargin}{-0.25in}{-0.5in}
  }
	{\end{changemargin}
}

% END CV DEFINITIONS

%%%% fancy header %%%%
% \pagestyle{fancy}
% \fancyhf{}% Clear header/footer
% \renewcommand{\headrulewidth}{0pt}
% \fancyhead[RO]{Last updated on \today}
% \begin{changemargin}{-0.5in}{-0.5in}
		% \begin{center}
            % \fancyhead[RO]{\textbf{Last updated on \today}}
		% \end{center}
	% \end{changemargin}
% \fancypagestyle{empty}{%
  % \fancyhf{}% Clear header/footer
  % \fancyhead[R]{Last updated on \today}% Your journal/note
% }


%%%%%%%%%%%%%%%%%%%%%%
\begin{document}

%%%% fancy header %%%%
% \pagestyle{fancy}
% \fancyhead[RO]{Last updated on \today}
% \fancypagestyle{empty}{%
% \fancyhf{}% Clear header/footer
% \renewcommand{\headrulewidth}{0pt}
% \fancyhead[R]{Last updated on \today}
% }

%%%% Name %%%%
\rightline{\footnotesize{Last updated on \today}}
\vspace{10pt}

\contact{Jeongwhan Choi}{\href{mailto:email@address.com}{jeongwhan.choi@yonsei.ac.kr}}{\href{https://jeongwhanchoi.com}{Homepage Link}}{\href{https://scholar.google.com/citations?user=3MNElkYAAAAJ&hl=en}{Google Scholar Link}}

%%%% Bio %%%%
% \header{Introduction}
% \begin{body}
%     \vspace{14pt}
%     I am currently a Ph.D. student at Yonsei University. I relocated to KAIST in February 2024 to continue my research with my advisor, Prof. Noseong Park, and currently I am staying at KAIST. My research interests span a variety of areas, including graph neural networks, recommender systems, spatio-temporal forecasting, and differential equations. I have developed graph-based deep learning methods inspired by differential equations in natural science, such as heat diffusion and reaction-diffusion equations. Recently, my focus on research has been extending the self-attention mechanism, the heart of Transformers, into the concept of graph signal processing.
% \end{body}

% \medskip
% %%%% Research Interest %%%%

% \header{Research Interest}

% \begin{body}
%     \vspace{14pt}
%     \begin{itemize}
%         \item Artificial Intelligence
%         \begin{itemize}
%             \item Graph Neural Networks
%             \item Recommendation (Collaborative Filtering and Sequential Recommendation)
%             \item Spatiotemporal Forecasting
%             \item Differential Equation-based Deep Learning
%             \item Transformers
%         \end{itemize}
%         \item Software Engineering
%         \begin{itemize}
%             \item Software Defect Prediction
%             \item AI-based Software Analytics
%         \end{itemize}
%     \end{itemize}
% \end{body}
% \medskip

%%%% Education %%%%
\header{Education}

\begin{body}
	\vspace{14pt}
	\textbf{Yonsei University, Republic of Korea} \hfill \emph{Sep 2020 -  Aug 2025} \\
	\emph{Ph.D. in Artificial Intelligence} \hfill \emph{Supervisors: Noseong Park and Sung-Bae Cho}\\
    \begin{itemize}
        \item Thesis Title: Empowering Deep Learning on Graphs: Tackling Over-smoothing and Over-squashing
        \item GPA: 4.21/4.3
    \end{itemize}
\end{body}

\begin{body}
	\vspace{14pt}
	\textbf{Jeonbuk National University, Republic of Korea} \hfill \emph{Mar 2016 -  Aug 2020} \\
	\emph{B.S. in Software Engineering}\\
    \item GPA: 3.98/4.5 (Presidential Award for Outstanding Undergraduate)
	% \begin{itemize} \itemsep -0pt  % reduce space between items
 %        \item \textit{magna cum laude}
	% \end{itemize}
\end{body}

\medskip
%%%% Work Experience %%%%
\header{Research Experience}
\begin{body}
	\vspace{14pt}
	\textbf{Ph.D. Candidate} \hfill \emph{Aug 2020 - Now} \\
	\emph{\href{https://sites.google.com/view/npark}{BigDyL}}, Yonsei University (Advisor: Prof. {\href{https://scholar.google.com/citations?user=VSuM3gYAAAAJ&hl=en}{Noseong Park} (affiliated with KAIST since 2024)})  \\
	% \begin{itemize} \itemsep -0pt  % reduce space between items
 %      \item  
 %  	\end{itemize}
	\textbf{Undergraduate Student Research Assistant} \hfill \emph{Jan 2020 - Aug 2020} \\
	\emph{\href{https://sites.google.com/view/aiselabjbnu}{AI \& SE Lab}}, Jeonbuk National University (Advisor: Prof. {\href{https://scholar.google.com/citations?user=BHue-MMAAAAJ&hl=en}{Duksan Ryu}})  \\
	% \begin{itemize} \itemsep -0pt  % reduce space between items
 %      \item  A Study on Quantitative Software Defect Prediction and Evaluation Techniques Based on Machine Learning
 %  	\end{itemize}
	\textbf{Undergraduate Student Research Assistant} \hfill \emph{Nov 2018 - Nov 2019} \\
	\emph{\href{https://sites.google.com/view/jbnussel/}{Software System and Engineering Laboratory (SSEL)}}, Jeonbuk National University (Advisor: Prof. {\href{https://scholar.google.com/citations?user=IN_HTKEAAAAJ&hl=en}{Suntae Kim}})  \\
	% \begin{itemize} \itemsep -0pt  % reduce space between items
 %      \item  Development of capability assessment evaluation algorithm for personalized self-study with Hanja-Chinese parallel.
 %  	\end{itemize}
\end{body}
\medskip


%%%% Publication %%%%

\header{Publications}

\begin{body}
	\vspace{14pt}
{\small * indicates equal contribution.}
	\vspace{14pt}

\textbf{Journal Papers}
\begin{enumerate}
    \item Yeon Uk Jeong, \textbf{Jeongwhan Choi}, Noseong Park, Jae Yong Ryu, and Yi Rang Kim, ``Predicting Drug-Drug Interactions: A Deep Learning Approach with GCN-Based Collaborative Filtering'', Artificial Intelligence in Medicine, 2025. [\RED{IF 6.1}] [\href{https://doi.org/10.1016/j.artmed.2025.103185}{Paper}]
    \item Yeon Uk Jeong, \textbf{Jeongwhan Choi}, Noseong Park, Jae Yong Ryu, and Yi Rang Kim, ``Predicting Drug-Drug Interactions: A Deep Learning Approach with GCN-Based Collaborative Filtering'', Artificial Intelligence in Medicine, 2025. [\RED{IF 6.1}] [\href{https://doi.org/10.1016/j.artmed.2025.103185}{Paper}]
    \item Taeyang Lee, \textbf{Jeongwhan Choi}, Inyeob Na, Insun Yoo, Sungil Woo, Kwang Jong Kim, Mikyung Park, Joonghwan Yang, Jeongguk Min, Seokwoo Lee, Noseong Park, Joonyoung Yang,``Graph-Based Representation Approach for Deep Learning of Organic Light-Emitting Diode Devices'', Advanced Intelligent System, 2024. [\RED{IF=6.8}][\href{https://onlinelibrary.wiley.com/doi/10.1002/aisy.202400598}{Paper}]
    \item \textbf{Jeongwhan Choi} and Noseong Park, ``Graph Neural Rough Differential Equations for Traffic Forecasting'', \emph{ACM Transactions on Intelligent Systems and Technology (TIST)}, 2023. [\href{https://dl.acm.org/doi/abs/10.1145/3604808}{paper}][\RED{IF=10.489}]
    \item Hwangyong Choi, \textbf{Jeongwhan Choi}, Jeehyun Hwang, Kookjin Lee, Dongeun Lee and Noseong Park, ``Climate Modeling with Neural Advection-Diffusion Equation'', \emph{Knowledge and Information Systems}, Jan. 2023. [\href{https://doi.org/10.1007/s10115-023-01829-2}{paper}] [\RED{IF=3.205(2021) Five year impact factor}]
    \item \textbf{Jeongwhan Choi} and Duksan Ryu, ``Bayesian Optimization Framework for Improved Cross-Version Defect Prediction'', \emph{KIPS Transactions on Software and Data Engineering (KTSDE)}, Vol. 10, No. 9, pp. 339-348, Sep. 2021.
    \item \textbf{Jeongwhan Choi}, Jiwon Choi, Duksan Ryu and Suntae Kim, ``Improved Prediction for Configuration Bug Report Using Text Mining and Dimensionality Reduction'', \emph{Journal of KIISE}, 2021, Vol. 48, No. 1, pp. 35-42.
    \item \textbf{Jeongwhan Choi}, Jiwoo Noh, and Suntae Kim, ``Prediction Techniques for Difficulty Level of Hanja Using Multiple Linear Regression'', \emph{The Journal of the Institute of Internet, Broadcasting and Communication}, vol. 19, no. 6, 2019.
    \item \textbf{Jeongwhan Choi}, ``Iceberg-Ship Classification in SAR Images Using Convolutional Neural Network with Transfer Learning'', \emph{Journal of Internet Computing and Services}, vol. 19, no. 4, pp. 35–44, 2018.
\end{enumerate}

\textbf{Conference Papers}
\begin{enumerate}
    \item Hyowon Wi, \textbf{Jeongwhan Choi}, and Noseong Park, ``Learning Advanced Self-Attention for Linear Transformers in the Singular Value Domain'', in International Joint Conference on Artificial Intelligence (\textit{IJCAI}), 2025. [\RED{Acceptance Rate 19.3\% (1042/5404)}]
    \item Youn-Yeol Yu*, \textbf{Jeongwhan Choi}*, Jaehyeon Park, Kookjin Lee, and Noseong Park,``PIORF: Physics-Informed Ollivier-Ricci Flow for Long–Range Interactions in Mesh Graph Neural Networks'', In International Conference on Learning Representations (ICLR), 2025. [\RED{Acceptance Rate 32.08\%}] [\href{https://openreview.net/forum?id=qkBBHixPow}{Paper}]
    \item Seonkyu Lim*, \textbf{Jeongwhan Choi}*, Jaehoon Lee, Noseong Park,``FrAug: Enhanced Fraud Detection in Interbank Transfers via Augmented Account Features'', In IEEE BigComp, 2025.
    \item Chaejeong Lee*, \textbf{Jeongwhan Choi}*, Hyowon Wi,  Sung-Bae Cho, and Noseong Park,``SCONE: A Novel Stochastic Sampling to Generate Contrastive Views and Hard Negative Samples for Recommendation'', In the 18th ACM International Conference on Web Search and Data Mining (WSDM), 2025. [\RED{Acceptance rate: 17.3\% (106/615)}]
    \item \textbf{Jeongwhan Choi}*, Hyowon Wi*, Jayoung Kim, Yehjin Shin, Kookjin Lee, Nathaniel Trask, Noseong Park,``Graph Convolutions Enrich the Self-Attention in Transformers!'', In Conference on Neural Information Processing Systems (\textit{NeurIPS}), 2024. [\RED{Paper Acceptance rate: 25.8\%}][\href{https://arxiv.org/abs/2312.04234}{arXiv}]
    \item Seonkyu Lim*, \textbf{Jeongwhan Choi}*, Sang-Ha Yoon, Shinhyuck Kang, Young-Min Kim, and Hyunjoong Kang, ``Bridging Dynamic Factor Models and Neural Controlled Differential Equations for Nowcasting GDP'', In Proceedings of the 33rd ACM International Conference on Information and Knowledge Management (\textit{CIKM}), 2024. [\RED{Applied Research Paper Acceptance rate: 32.59\% (103/316)}]
    \item \textbf{Jeongwhan Choi}, Sumin Park, Hyowon Wi,  Sung-Bae Cho, and Noseong Park, ``PANDA: Expanded Width-Aware Message Passing Beyond Rewiring'', in International Conference on Machine Learning (\textit{ICML}), 2024. [\RED{Acceptance Rate 27.5\% (2610/9473)}][\href{https://proceedings.mlr.press/v235/choi24f.html}{Paper}][\href{https://arxiv.org/abs/2406.03671}{arXiv}]
    \item Jayoung Kim, Yehjin Shin, \textbf{Jeongwhan Choi}, Hyowon Wi, Noseong Park, ``Polynomial-based Self-Attention for Table Representation Learning'',  in International Conference on Machine Learning (\textit{ICML}), 2024. [\RED{Acceptance Rate 27.5\% (2610/9473)}][\href{https://arxiv.org/abs/2312.07753}{arXiv}]
    \item Seoyoung Hong, \textbf{Jeongwhan Choi}, Yeon-Chang Lee, Srijan Kumar, Noseong Park, ``SVD-AE: Simple Autoencoders for Collaborative Filtering'', in International Joint Conference on Artificial Intelligence (\textit{IJCAI}), 2024. [\RED{Acceptance Rate 14.00\% (791/5651)}][\href{https://arxiv.org/abs/2405.04746}{arXiv}]
    \item Youn-Yeol Yu, \textbf{Jeongwhan Choi}, Woojin Cho, Kookjin Lee, Nayong Kim, Kiseok Chang, ChangSeung Woo, Ilho Kim, SeokWoo Lee, Joon Young Yang, Sooyoung Yoon, and Noseong Park, ``Learning Flexible Body Collision Dynamics with Hierarchical Contact Mesh Transformer'',  in \textit{ICLR}, 2024. [\RED{Acceptance Rate  30.81\% (2250/7304)}][\href{https://arxiv.org/abs/2312.12467}{arXiv}]
    \item Yehjin Shin*, \textbf{Jeongwhan Choi}*, Hyowon Wi, Noseong Park, ``An Attentive Inductive Bias for Sequential Recommendation beyond the Self-Attention'',  in \textit{AAAI}, 2024. [\RED{Acceptance Rate 23.75\% (2342/12100)}] [\href{https://arxiv.org/abs/2312.10325}{arXiv}]
    \item Seonkyu Lim, Jaehyeon Park, Seojin Kim, Hyowon Wi, Haksoo Lim, Jinsung Jeon, \textbf{Jeongwhan Choi}, and Noseong Park, ``Long-term Time Series Forecasting based on Decomposition and Neural Ordinary Differential Equations'', In \emph{IEEE International Conference on Big Data (Big Data)}, 2023.
    \item \textbf{Jeongwhan Choi} and Duksan Ryu, ``QoS-Aware Graph Contrastive Learning for Web Service Recommendation'', In \emph{Proceedings of the 30th Asia-Pacific Software Engineering Conference (APSEC 2023)}, 2023.
    \item \textbf{Jeongwhan Choi}, Seoyoung Hong, Noseong Park and Sung-Bae Cho, ``GREAD: Graph Reaction-Diffusion Networks'', In \emph{Proceedings of the 40th International Conference on Machine Learning (ICML)}, 2023. [\href{http://proceedings.mlr.press/v202/choi23a}{paper}][\href{https://github.com/jeongwhanchoi/GREAD}{code}][\RED{Paper Acceptance rate: 27.94\% (1,827/6,538)}]
    \item \textbf{Jeongwhan Choi}, Seoyoung Hong, Noseong Park and Sung-Bae Cho, ``Blurring-Sharpening Process Models for Collaborative Filtering'', In \emph{Proceedings of the 46th ACM Conference on Research and Development in Information Retrieval (SIGIR)}, 2023. [\href{https://arxiv.org/abs/2211.09324}{paper}][\href{https://github.com/jeongwhanchoi/bspm}{code}][\RED{Paper Acceptance rate: 20.1\% (165/822)}]
    \item \textbf{Jeongwhan Choi} and Duksan Ryu, ``Graph Convolution-based Collaborative Filtering for Web Service QoS Ranking'', In \emph{Proceedings of the 25th Korea Conference on Software Engineering (KCSE 2023)}, 2023, pp. 58-67.
    \item Seoyoung Hong, Heejoo Shin, \textbf{Jeongwhan Choi}, and Noseong Park, ``Prediction-based One-shot Dynamic Parking Pricing'', In \emph{Proceedings of the 31st ACM International Conference on Information and Knowledge Management (CIKM)}, 2022.[\href{https://arxiv.org/abs/2208.14231}{paper}][\href{https://github.com/jeongwhanchoi/one-shot-optimization}{code}]
    \item \textbf{Jeongwhan Choi}, Hwangyong Choi, Jeehyun Hwang and Noseong Park, ``Graph Neural Controlled Differential Equations for Traffic Forecasting'', In \emph{AAAI}, 2022. [\href{https://ojs.aaai.org/index.php/AAAI/article/download/20587/20346}{paper}][\href{https://github.com/jeongwhanchoi/STG-NCDE}{code}][\RED{Regular Paper Acceptance rate: 14.2\% (1,161/8,198)}] [\RED{Overall Acceptance rate: 15.2\% (1,370/9,020)}]
    \item Taeyong Kong, Taeri Kim, Jinsung Jeon, \textbf{Jeongwhan Choi}, Yeon-Chang Lee, Noseong Park and Sang-Wook Kim, ``Linear, or Non-Linear, That is the Question!'', In \textit{Proceedings of the 15th ACM International Web Search and Data Mining Conference (WSDM)}, 2022.  [\href{https://arxiv.org/abs/2111.07265}{paper}][\href{https://github.com/jeongwhanchoi/HMLET}{code}] [\RED{Regular Paper Acceptance rate: 15.8\% (80/505)}] [\RED{Overall Acceptance Rate: 18\%  (315/1,765)} ]
    \item \textbf{Jeongwhan Choi} and Duksan Ryu, ``Self-Supervised Learning Using Feature Subsets of Software Defect Data'',  In \emph{Proceedings of the Korea Software Congress (KSC)}, Dec. 2021, pp.203-205.
    \item \textbf{Jeongwhan Choi}, Jinsung Jeon, and Noseong Park, ``LT-OCF: Learnable-Time ODE-based Collaborative Filtering'', In \emph{Proceedings of the 30th ACM International Conference on Information and Knowledge Management (CIKM)}, 2021. [\href{https://arxiv.org/pdf/2108.06208.pdf}{paper}] [\href{https://github.com/jeongwhanchoi/LT-OCF}{code}]  [\RED{Regular paper acceptance rate: 21.7\% (271/1,251)}]  [\RED{Overall Acceptance rate: 22\% (1,101/4,989)]}
    \item \textbf{Jeongwhan Choi} and Duksan Ryu, ``Bayesian Optimization Framework for Cross-Version Defect Prediction'', In \emph{Proceedings of the 23rd Korea Conference on Software Engineering (KCSE 2021)}, 2021, pp. 63-72.  [\RED{Best Paper}]
    \item \textbf{Jeongwhan Choi} and Duksan Ryu, ``A Study on the Applicability of Transfer Learning Techniques for Cross-Project Defect Regression'', In \emph{Proceedings of the Korea Software Congress (KSC)}, 2020, pp. 150 - 152.
    \item \textbf{Jeongwhan Choi}, Duksan Ryu, and Suntae Kim, ``Comparative Study of Transfer Learning Models for Cross-Project Automotive Software Defect Prediction'', In \emph{Proceedings of the Korea Computer Congress (KCC)}, 2020, pp. 257–259.
    \item \textbf{Jeongwhan Choi}, Jiwon Choi, Duksan Ryu, and Suntae Kim, ``Prediction for Configuration Bug Report Using Text Mining'', In \emph{Proceedings of the 22nd Korea Conference on Software Engineering (KCSE 2020)}, 2020, pp. 350–357.
    \item Seounghan Song, \textbf{Jeongwhan Choi}, Mingu Kang, and Cheoljung Yoo, ``A Software Module That Analyzes the Relationship Between Headline and Content of the Web Article: CHIMERA'', in \emph{Proceedings of the 2019 KIIT DCS Summer Conference}, 2019, vol. 14, pp. 437–440.
\end{enumerate}

\textbf{Abstracts and Contributions to Peer-reviewed Workshops}
\begin{enumerate}
    \item Jinsung Jeon, Jaehyeon Park, Sewon Park, \textbf{Jeongwhan Choi}, Minjung Kim, and Noseong Park, ``Possibility for Proactive Anomaly Detection'', I Can't Believe It's Not Better: Challenges in Applied Deep Learning (ICBINB) Workshop at ICLR, 2025. [\href{https://openreview.net/forum?id=w63aCqNRFp}{Paper}]
    \item Seonkyu Lim*, \textbf{Jeongwhan Choi}*, Noseong Park,``FraudCenGCL: Enhancing Fraud Detection via Dual-View Graph Contrastive Learning with Account Centrality Features'', In Social Impact: Bridging Innovations in Finance, Social Media, and Crime Prevention Workshop at AAAI 2025.
    \item Seonkyu Lim*, \textbf{Jeongwhan Choi}*, Jaehoon Lee, Noseong Park,``FrAug: Enhanced Fraud Detection in Interbank Transfers via Augmented Account Features'', In AI4TS Workshop at AAAI 2025.
    \item Seonkyu Lim*, \textbf{Jeongwhan Choi}*, Jaehoon Lee, Noseong Park,``Enhanced Fraud Detection in Bank Transfers via Augmented Account Features'', ACM ICAIF Workshop on Foundation Models for Time Series: Exploring New Frontiers (FM4TS), 2024. [\RED{Accepted for oral presentation}]
    \item Jaehoon Lee, Chan Kim, Gyumin Lee, Haksoo Lim, \textbf{Jeongwhan Choi}, Kookjin Lee, Dongeun Lee, Sanghyun Hong, and Noseong Park, ``HyperNetwork Approximating Future Parameters for Time Series Forecasting under Temporal Drifts'', \emph{NeurIPS 2023 Workshop on Distribution Shifts (DistShift)}, 2023.
    \item Jeehyun Hwang, \textbf{Jeongwhan Choi}, Hwangyong Choi, Kookjin Lee, Dongeun Lee, and Noseong Park, ``Climate Modeling with Neural Diffusion Equations'', In \emph{Proceedings of the 21st IEEE International Conference on Data Mining (ICDM)}, 2021.  [\href{https://arxiv.org/abs/2111.06011}{paper}] [\href{https://github.com/jeongwhanchoi/Neural-Diffusion-Equation}{code}] [\RED{Regular paper acceptance rate: 9.9\%  (98/990)}] [\RED{Overall Acceptance Rate: 20\%  (198/990)}]
\end{enumerate}

\textbf{Preprint Papers}
\begin{enumerate}
    \item Jeongeun Lee, Seongku Kang, Won-Yong Shin, \textbf{Jeongwhan Choi}, Noseong Park, and Dongha Lee, "Graph Signal Processing for Cross-Domain Recommendation", arXiv preprint arXiv: Arxiv-2407.12374. [\href{https://arxiv.org/abs/2407.12374}{arXiv}]
    \item \textbf{Jeongwhan Choi}*, Hyowon Wi*, Chaejeong Lee, Sung-Bae Cho, Dongha Lee, Noseong Park, ``RDGCL: Reaction-Diffusion Graph Contrastive Learning for Recommendation'',  arXiv preprint arXiv: Arxiv-2312.16563, 2023. [\href{https://arxiv.org/abs/2312.16563}{arXiv}]
    \item Jaehoon Lee, Chan Kim, Gyumin Lee, Haksoo Lim, \textbf{Jeongwhan Choi}, Kookjin Lee, Dongeun Lee, Sanghyun Hong, and Noseong Park, ``Time Series Forecasting with Hypernetworks Generating Parameters in Advance'', \emph{arXiv preprint arXiv: Arxiv-2211.12034}, 2022. [\href{https://arxiv.org/abs/2211.12034}{paper}]
\end{enumerate}
\end{body}

\medskip
% \pagebreak
%%%% Honors & Awards %%%%

\header{Awards \& Scholarships}
\begin{body}
	\vspace{14pt}
\award{Outstanding Reviewer (Top 10\%), KDD 2025 (February Cycle).} {Jun 2025}
\award{Merit Academic Paper Award, Yonsei University (Runner-up paper in Dept. of Artificial Intelligence) \href{https://graduate.yonsei.ac.kr/graduate/board/news.do?mode=view&articleNo=220032&article.offset=0&articleLimit=10}{[link]}} {Feb 2025} 
\award{Outstanding Reviewer (Top 10\%), KDD 2025 (August Cycle).} {Dec 2024}
\award{Qualcomm Innovation Fellowship Winner, Qualcomm Technologies Inc. (4M KRW, approx. 2.8K USD) \href{https://www.qualcomm.com/research/university-relations/innovation-fellowship/2024-south-korea}{[link]}} {Dec 2024} 
\award{Top Reviewer Award, LoG 2024. \href{https://x.com/LogConference/status/1862602407395697123}{[link]}} {Nov 2024}
\award{Top Reviewer Award, NeurIPS 2024. \href{https://neurips.cc/Conferences/2024/ProgramCommittee\#top-reviewers}{[link]}} {Nov 2024}
\award{First Prize of Idea Competition, Korea Financial Telecommunications \& Clearning Institute. \href{https://drive.google.com/file/d/1--cUGRX6rB9rc2phbMeIJ7BdHRNCj_au/view?usp=sharing}{[link]}} {Oct 2024}
\award{Merit Academic Paper Award, Yonsei University (Runner-up paper in Dept. of Artificial Intelligence) \href{https://graduate.yonsei.ac.kr/graduate/board/news.do?mode=view&articleNo=208151&article.offset=0&articleLimit=10}{[link]}} {Jul 2024} 
\award{Best Paper Award, Yonsei University (Best paper in Dept. of Artificial Intelligence) \href{https://graduate.yonsei.ac.kr/graduate/board/news.do?mode=view&articleNo=181558&article.offset=0&articleLimit=10}{[link]}} {Jan 2024} 
\award{Graduate Student Research Assistantship (GSRA) Scholarship, Yonsei University}{May 2023}
\award{Academic Research Fellowship (ARF), Brain Korea 21 (BK21) Program, Yonsei University}{Jan 2023}
\award{Graduate Student Research Assistantship (GSRA) Scholarship, Yonsei University}{Nov 2022}
\award{Academic Research Fellowship (ARF), Brain Korea 21 (BK21) Program, Yonsei University}{Sep 2022}
\award{Innovation Award, Yonsei University (Best paper in Dept. of Artificial Intelligence) \href{https://www.yonsei.ac.kr/ocx/news.jsp?mode=view&ar_seq=20220708141917269049&sr_volume=632&list_mode=list&sr_site=S&pager.offset=0&sr_cates=29}{[link]}}{Jul 2022}
\award{Graduate Student Research Assistantship (GSRA) Scholarship, Yonsei University}{Apr 2022}
\award{Best Paper Awards in the 23rd Korea Conference on Software Engineering (KCSE 2021)}{Feb 2021}
\award{Presidential Award for Outstanding Undergraduate, Jeonbuk National University}{Dec 2019}
\award{Best Paper Awards in Dept. of Software Engineering, Jeonbuk National University}{Dec 2019}
\award{Best Paper Awards, Korean Institute of Information Technology}{Jun 2019}
\award{The National Scholarship for Science and Engineering, KOSAF(Korea Student Aid Foundation)}{2018-2019}
    \begin{itemize} \itemsep -0pt  % reduce space between items
      \item This scholarship supports undergraduates with strong academic performance in science and engineering, with the purpose of developing future leaders in these fields.
  	\end{itemize}
\award{Academic Excellent Scholarship}{2016-2019}
\end{body}

\medskip
% \pagebreak
%%%% Talks %%%%
\header{Talks}

\begin{body}
	\vspace{14pt}
\award{Talk on Top-conference session, Korea Software Congress (KSC 2023)} {Dec 2023}
\award{Talk on 1st Seminar held by Graph User Group (GUG)}{Jun 2023}
\award{Talk on Top-conference session, Korea Computer Congress (KCC 2023) [\href{https://www.dropbox.com/s/34h6pmr7ftdiuzr/BSPM-KCC23.pptx?dl=0}{slides}]}{Jun 2023}
\award{Talk on 2023 KSIAM AI Winter School, held by Korean Society for Industrial and Applied Mathematics (KSIAM)[\href{https://www.dropbox.com/s/p4sd5h40hcuxcob/KSIAM23-Tutorial-ODE-RecSys.pdf?dl=0}{slides}][\href{https://ksiam.org/Conference/ConferenceView.asp?AC=3&CODE=CD20230101&CpPage=\#CONF}{website}]}{Feb 2023}
\award{Talk on 1st Workshop on AI held by Yonsei Univ. [\href{https://www.dropbox.com/s/9au5xx13qa2l529/AAAI22_workshop.pdf?dl=0}{slides}][\href{https://www.dropbox.com/s/pibzd51d76zy907/AAAI22-Yonsei_AI_Workshop.pdf?dl=0}{poster}]}{Oct 2022}
% \award{Poster presentation for AIGS Symposium 2022 held at the COEX Grand Ballroom [\href{https://www.dropbox.com/s/gfjsizak9s4cn9o/AAAI22-AIGS.pdf?dl=0}{poster}]}{Aug 2022}
\award{Talk on Top-conference session, Korea Computer Congress (KCC 2022) [\href{https://www.dropbox.com/s/22d9d92ns8uv9qw/AAAI22_KCC22.pdf?dl=0}{slides}]}{Jul 2022}
\award{Tutorial on Korea Artificial Intelligence Association (KAIA)}{Nov 2021}
	\begin{itemize} \itemsep -0pt  % reduce space between items
        \item Topic: ``Graph-based Collaborative Filtering and Neural ODEs''
		\item This talk is part of a tutorial called ``Deep Learning Inspired by Differential Equation'' [\href{https://www.dropbox.com/s/1xn8xhd6llmhblz/%5BKAIA2021%5DTutorial-LT-OCF.pdf?dl=0}{slides}].
  	\end{itemize}
\end{body}

\medskip

%%%% Service %%%%
\header{Academic Services}

\begin{body}
	\vspace{14pt}
    Conference Organizer
    \begin{itemize}
        \item Website Chair: LoG 2025
        \item Web Chair: CIKM 2025
    \end{itemize}
   Reviewer or Program Committee Member for Conference
    \begin{itemize}
        \item ICANN 2025
        \item ECAI 2025
        \item SIGIR 2025
        \item ICLR 2025
        \item AISTATS 2025
        \item ICLR 2025
        \item NeurIPS 2024 (\RED{Top 8\% Reviewer Award}), 2025 (Benchmark \& Datasets and Position Paper Tracks)
        \item AAAI 2024, 2025
        \item IJCAI 2024, 2025
        \item WSDM 2025, 2026
        \item CIKM 2024, 2025 (Full paper and short paper tracks)
        \item KDD 2023, 2024, 2025 August Track (\RED{Outstanding Reviewer}), 2025 February Track (\RED{Outstanding Reviewer}), 2026 First Cycle
        \item SDM 2024
        \item LoG 2022, 2023, 2024 (\RED{Top Reviewer Award}), 2025
        \item ICDM 2021, 2022
        \item AI4TS workshop at SDM 2025 and AAAI 2025
        \item 4th Workshop on Graphs and More Complex Structures for Learning and Reasoning (GCLR) colocated with AAAI 2024
    \end{itemize}
    Reviewer for Journal
    \begin{itemize}
        \item Expert Systems With Applications (ESWA) – 1 time in 2025
        \item IEEE Transactions on Emerging Topics in Computing – 1 time in 2025
        \item Transactions on Machine Learning Research (TMLR) – 4 times in 2025
        \item Knowledge-Based Systems – 3 times in 2025
        \item Applied Economics Letters – 1 time in 2024
        \item Neurocomputing – 2 times in 2024
        \item IEEE Transactions on Image Processing (TIP) – 1 time in 2024
        \item IEEE Transactions on Knowledge and Data Engineering (TKDE) – 1 time in 2023, 1 time in 2024
        \item Journal of Intelligent \& Fuzzy Systems – 1 time in 2024
        \item Applied Soft Computing – 6 times in 2024, 9 times in 2025
        \item Applied Artificial Intelligence – 2 times in 2023
        \item IEEE Transactions on Intelligent Transportation Systems (TIST) – 1 time in 2022, 1 time in  2025
    \end{itemize}
   Reviewer for Workshop
    \begin{itemize}
        \item AI4TS workshop at SDM 2025 and AAAI 2025
        \item 4th Workshop on Graphs and More Complex Structures for Learning and Reasoning (GCLR) colocated with AAAI 2024
    \end{itemize}
\end{body}

\medskip
% \pagebreak

%%%% Patent %%%%
\header{Patents}

\begin{body}
	\vspace{14pt}

\presentation{[Patent Application] Electronic Apparatus for Providing Recommendation Information and Operating Method Thereof, Noseong Park, Yehjin Shin, \textbf{Jeongwhan Choi}, Hyowon Wi, US Patent (Application No. 18/931,310). 2024.10.30}{Oct 2024}	
	
\presentation{[Patent Application] 추천 정보를 제공하는 전자 장치 및 그 동작 방법, Noseong Park, Yehjin Shin, \textbf{Jeongwhan Choi}, Hyowon Wi, KR Patent (Application No. 10-2024-0136316). 2024.10.08}{Oct 2024}	


\presentation{[Patent Application] 이상 탐지 시스템 및 그 방법, Sewon Park, Minjung Kim, Noseong Park, Jinsung Jeon, \textbf{Jeongwhan Choi}, Jaehyun Park, KR Patent (Application No. 10-2024-006519). 2024.01.06 }{Jan 2024}

\presentation{[Patent Application] Apparatus and Method for Processing Spatiotemporal Data Based on Graph Neural Controlled Differential Equations, Noseong Park, \textbf{Jeongwhan Choi}, Jeehyun Hwang, Hwangyong Choi, US Patent (Application No. 18/085,109). 2022.12.20 \href{https://patents.google.com/patent/US20230186105A1/en}{[link]}}{Dec 2022}	

\presentation{[Patent Application] 그래프 신경 제어 미분 방정식 기반의 시공간 데이터 처리 장치 및 방법, Noseong Park, \textbf{Jeongwhan Choi}, Jeehyun Hwang, Hwangyong Choi, KR Patent (Application No. 10-2022-0151819). 2022.11.14 \href{https://patents.google.com/patent/KR20240073179A/en}{[link]}}{Nov 2022}

\presentation{[Patent Application] 学習時間常微分方程式基盤の協業フィルタリング推薦装置及び方法, Noseong Park, \textbf{Jeongwhan Choi}, Jinsung Jeon, JP Patent (Application No. 2021-215162). 2021.12.28}{Dec 2021}	

\presentation{[Patent Application] Apparatus and Method for Collaborative Filtering Based on Learnable-Time Ordinary Differential Equation, Noseong Park, \textbf{Jeongwhan Choi}, Jinsung Jeon, US Patent (Issued No.	17/563726). 2021.12.28 \href{https://patents.google.com/patent/US20230186105A1/en}{[link]}}{Dec 2021}	

\presentation{[Patent Application] 학습시간 상미분 방정식 기반의 협업 필터링 추천 장치 및 방법, Noseong Park, \textbf{Jeongwhan Choi}, Jinsung Jeon, KR Patent (Issued No. 10-2021-0177928). 2021.12.13 \href{http://kpat.kipris.or.kr/kpat/biblioa.do?method=biblioFrame}{[link]}}{Dec 2021}

\presentation{[Patent Application] 회귀 분석을 이용한 한자 난이도 측정 장치 및 방법, Suntae Kim, \textbf{Jeongwhan Choi}, Jiwoo Noh, KR Patent (KR Patent No. 10-2019-0141339 ). 2019.11.07 \href{https://doi.org/10.8080/1020190141339}{[link]}}{Nov 2019}	

\presentation{[Granted Patent] 회귀 분석을 이용한 한자 난이도 측정 장치 및 방법, Suntae Kim, \textbf{Jeongwhan Choi}, Jiwoo Noh, KR Patent (Application No. 10-2321045 ). 2021.10.28 \href{https://doi.org/10.8080/1020190141339}{[link]}}{Oct 2021}
  
\end{body}
    
\medskip

% \clearpage
%%%% Refe %%%%
\header{References}

\begin{body}
	\vspace{14pt}
    \begin{itemize}
        \item Noseong Park  (noseong@kaist.ac.kr)
        \item Sung-Bae Cho  (sbcho@yonsei.ac.kr)
        \item Duksan Ryu (duksan.ryu@jbnu.ac.kr)
        \item Suntae Kim (stkim@jbnu.ac.kr)
    \end{itemize}
\end{body}
%%%% Certifications %%%%

% \header{Certifications}

% \begin{body}
% 	\vspace{14pt}

%   %% UMLC Jeju  %%
%   	\labtitle{University Machine Learning Camp in Jeju}{Jeju University}{Aug 2020}
%   \labdescription {
%   	\begin{itemize} \itemsep -0pt  % reduce space between items
%       \item \href{https://drive.google.com/file/d/1lV5w4cdCMVgFnBGu2sHeTigXFEZCX3Sr/view}{See credential}
%   	\end{itemize}
%   }
%   %% IBM Blockchain Foundation for Developers  %%
%   	\labtitle{IBM Blockchain Foundation for Developers }{Coursera}{Feb 2018 - Present}
%   \labdescription {
%   	\begin{itemize} \itemsep -0pt  % reduce space between items
%       \item License 5MMQUBFWE2K3 (\href{https://www.credly.com/badges/a9f84e03-9bc0-462e-8dd2-66ed9c7878c0/linked_in_profile}{See credential})
%   	\end{itemize}
%   }
  
% %% Machine Learning Engineer Nanodegree %%
%   	\labtitle{Machine Learning Engineer Nanodegree}{Udacity}{Jan 2018 - Present}
%   \labdescription {
%   	\begin{itemize} \itemsep -0pt  % reduce space between items
%         \item \href{https://graduation.udacity.com/confirm/AN2EMAQD}{See credential}
%   	\end{itemize}
%   }

% %% Machine Learning%%
%   	\labtitle{Machine Learning}{Coursera}{July 2017 - Present}
%   \labdescription {
%   	\begin{itemize} \itemsep -0pt  % reduce space between items
%       \item License EEYYGQPCFLN7 (\href{https://www.coursera.org/account/accomplishments/verify/EEYYGQPCFLN7}{See credential})
%   	\end{itemize}
%   }
  


% \end{body}

% \medskip

% %%%% Skills %%%%
% \header{Skills}

% \begin{body}
% 	\vspace{14pt}
% %% Tools & Technologies %%
%   	\labtitle{Tools \& Technologies}{}{}
%   \labdescription {
%   	\begin{itemize} \itemsep -0pt  % reduce space between items
%       \item PyTorch, TensorFlow, Python
%       \item Java, C/C++, R, LaTeX, VBA, Unified Modeling Language
%       \item Android, Matlab, Git, RSA
%       \item MySQL, Tomcat, JSP, HTML, Javascript, JUnit
%   	\end{itemize}
%   }
%   %% Industry Knowledge. %%
%   	\labtitle{Skills \& Knowledge}{}{}
%   \labdescription {
%   	\begin{itemize} \itemsep -0pt  % reduce space between items
%       \item Artificial Intelligence, Graph Neural Networks, Neural Ordinary Differential Equations (Neural ODEs), Recommender Systems, Transformers, Time-series Forecasting, Spatio-temporal Forecasting, Software Defect Prediction
%       \item Software Engineering, Object Oriented Programming, Design Pattern, Compiler, Software Testing (Static Analysis)
%       \item Text Mining, Image Processing for SAR Image
%       \item ARM Cortex-M3, ESP-8266
%   	\end{itemize}
%   }

% \end{body}

% \medskip
% \clearpage
% %%%% Projects %%%%

% \header{Projects (From 2017 to 2019)}
% \begin{body}
% 	\vspace{14pt}
% \textit{The list below includes projects conducted in class, personal projects, and projects conducted during undergraduate research internships. These are all projects I worked on as an undergraduate student.}\\
% 	\vspace{14pt}
% \project{Prediction for Configuration Bug Report Using Text Mining}{Nov 2019 - Dec 2019}
% \begin{itemize} \itemsep -0pt  % reduce space between items
%       \item  The purpose of this project is to predict the configuration bug reports using machine learning techniques and NLP.
%   	\end{itemize}
% \project{Development of capability assessment evaluation algorithm for personalized self-study with Hanja-Chinese parallel}{Dec 2018 - Expected Nov 2019}
% \begin{itemize} \itemsep -0pt  % reduce space between items
%       \item  The purpose of this project is to solve the problems of existing Hanja character difficulty selection method.
%       \item It includes the technique for measuring the difficulty of Hanja characters using artificial intelligence.
%       \item It also covers personalized learning induction technique using a clustering model.
%   	\end{itemize}
% \project{Stock Price Prediction Model Based LSTM to Maximize Return on Investment}{Oct 2018 - Dec 2018}
% \begin{itemize} \itemsep -0pt  % reduce space between items
%       \item  The purpose of this project is to predict the long-term stock flow based on the AI prediction model and to derive meaningful ROI.
%       \item This project has a paper which is not submitted.
%       \item \href{https://drive.google.com/file/d/1Fp3WmoBpHpYU13fNWS1vbQSzmBCR3MIu/view}{See project}
%   	\end{itemize}
% \project{Advanced Lane Finding Project}{Jul 2018}
% \begin{itemize} \itemsep -0pt  % reduce space between items
%       \item  The goal is to find the lane line using advanced image processing techniques.
%       \item \href{https://github.com/jeongwhanchoi/CarND-Advanced-Lane-Lines}{See project}
%   	\end{itemize}
% \project{Vehicle Detection}{Jul 2018}
% \begin{itemize} \itemsep -0pt  % reduce space between items
%       \item  The software pipeline using computer vision algorithms to detect vehicles in videos.
%       \item \href{https://github.com/jeongwhanchoi/CarND-Vehicle-Detection}{See project}
%   	\end{itemize}
% \project{Clone Driving  Behavior}{Jun 2018}
% \begin{itemize} \itemsep -0pt  % reduce space between items
%       \item  The goal is to clone driving behavior via the CNN model.
%       \item \href{https://github.com/jeongwhanchoi/CarND-Behavioral-Cloning}{See project}
%   	\end{itemize}
% \project{Recipe Assistant App}{Apr 2018 - Jun 2018}
% \begin{itemize} \itemsep -0pt  % reduce space between items
%       \item  The recipe assistant app which helps people to cook an easy way. This app was implemented by Android and improved by several design patterns.
%       \item \href{https://github.com/jeongwhanchoi/recipe-assistant-app}{See project}
%   	\end{itemize}
% \project{Tic-Tac-Toe game for LPC 1768}{Jun 2018}
% \begin{itemize} \itemsep -0pt  % reduce space between items
%       \item  This project is the Tic-Tac-Toe Game using ARM Cortex-M3(LPC 1768)
%       \item \href{https://github.com/jeongwhanchoi/tic-tac-toe-lpc1768}{See project}
%   	\end{itemize}
% \project{Lane Finding Project}{Mar 2018}
% \begin{itemize} \itemsep -0pt  % reduce space between items
%       \item  The goal is to find the lane lines on the road.
%       \item \href{https://github.com/jeongwhanchoi/CarND-LaneLines}{See project}
%   	\end{itemize}
% \project{Iceberg Classifier}{Jan 2018}
% \begin{itemize} \itemsep -0pt  % reduce space between items
%       \item  The goal is to create an image classification model that finds icebergs among SAR images collected by satellites. This project has a paper published in JICS.
%       \item \href{https://github.com/jeongwhanchoi/MLND-Capstone-Project}{See project}
%   	\end{itemize}
% \project{Helicopter Battle Game}{Apr 2017 - Jul 2017}
% \begin{itemize} \itemsep -0pt  % reduce space between items
%       \item  This Java project includes the software engineering knowledges such as objective-oriented design, design patterns and so on.
%       \item \href{https://github.com/jeongwhanchoi/helicopter_battle}{See project}
%   	\end{itemize}
% \project{Smart Mailbox}{Sep 2017 - Dec 2017}
% \begin{itemize} \itemsep -0pt  % reduce space between items
%       \item  The smart mailbox project played a role in helping people who forget offline mails through notifications to mobile phones.
%       \item \href{https://github.com/jeongwhanchoi/Smart-Mailbox}{See project}
%   	\end{itemize}	
% \end{body}


\medskip


\end{document}


