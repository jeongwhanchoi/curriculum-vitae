\documentclass[10pt]{article}
\usepackage[a4paper, total={6in, 8in}]{geometry}
\usepackage{fullpage}
\usepackage{amsmath}
\usepackage{amssymb}
\usepackage[usenames,dvipsnames]{xcolor}
\usepackage[]{hyperref}
\usepackage{kotex}
\usepackage[]{fancyhdr}
\hypersetup{
    colorlinks=true,
    citecolor=black,
    filecolor=black,
    linkcolor=NavyBlue,
    urlcolor=NavyBlue,
    breaklinks=true
}
\urlstyle{same}

\leftmargin=0.25in
\oddsidemargin=0.25in
\textwidth=6.0in
\topmargin=-0.25in
\textheight=9.25in

\raggedright

\pagenumbering{gobble} % suppresses page numbering. To stop suppression, change to 'arabic'.

\def\bull{\vrule height 0.8ex width .7ex depth -.1ex }


% DEFINITIONS FOR CV - defining some features to simplify template

\newenvironment{changemargin}[2]{
  \begin{list}{}{
    \setlength{\topsep}{0pt}
    \setlength{\leftmargin}{#1}
    \setlength{\rightmargin}{#2}
    \setlength{\listparindent}{\parindent}
    \setlength{\itemindent}{\parindent}
    \setlength{\parsep}{\parskip}
  }
  \item[]}{\end{list}
}

\newcommand{\lineover}{
	\begin{changemargin}{-0.05in}{-0.05in}
		\vspace*{-8pt}
		\hrulefill \\
		\vspace*{-2pt}
	\end{changemargin}
}

\newcommand{\header}[1]{
	\begin{changemargin}{-0.5in}{-0.5in}
		\scshape{#1}\\
  	\lineover
	\end{changemargin}
}

\newcommand{\contact}[5]{
	\begin{changemargin}{-0.5in}{-0.5in}
		\begin{center}
			{\Large \scshape {#1}}\\
      {#2} \\  {#3} \\ {#4} \\ {#5}
		\end{center}
	\end{changemargin}
}

\newcommand{\labdescription}[1]{
	\begin{changemargin}{0.15in}{0.15in}
    \smallskip
		{#1}
    \medskip
	\end{changemargin}
}

\newcommand{\labtitle}[3]{
	\textbf{#1}, \emph{#2} \hfill \emph{#3}\\
}

\newcommand{\presentation}[2]{
	{#1} \hfill \emph{#2}\\ \bigskip
}

\newcommand{\award}[2]{
	{#1} \hfill \emph{#2}\\ \medskip
}

\newcommand{\society}[2]{
	{#1} \hfill \emph{#2}\\ \medskip
}

\newcommand{\project}[2]{
	{#1} \hfill \emph{#2}\\ \medskip
}
\newcommand{\RED}[1]{\textcolor{red}{#1}}
\newcommand{\BLUE}[1]{\textcolor{blue}{#1}}

\newenvironment{body} {
	\vspace*{-16pt}
	\begin{changemargin}{-0.25in}{-0.5in}
  }
	{\end{changemargin}
}

% END CV DEFINITIONS

%%%% fancy header %%%%
% \pagestyle{fancy}
% \fancyhf{}% Clear header/footer
% \renewcommand{\headrulewidth}{0pt}
% \fancyhead[RO]{Last updated on \today}
% \begin{changemargin}{-0.5in}{-0.5in}
		% \begin{center}
            % \fancyhead[RO]{\textbf{Last updated on \today}}
		% \end{center}
	% \end{changemargin}
% \fancypagestyle{empty}{%
  % \fancyhf{}% Clear header/footer
  % \fancyhead[R]{Last updated on \today}% Your journal/note
% }


%%%%%%%%%%%%%%%%%%%%%%
\begin{document}

%%%% fancy header %%%%
% \pagestyle{fancy}
% \fancyhead[RO]{Last updated on \today}
% \fancypagestyle{empty}{%
% \fancyhf{}% Clear header/footer
% \renewcommand{\headrulewidth}{0pt}
% \fancyhead[R]{Last updated on \today}
% }

%%%% Name %%%%
\rightline{\footnotesize{Last updated on \today}}
\vspace{10pt}

\contact{Jeongwhan Choi}{\href{mailto:email@address.com}{jeongwhan.choi@yonsei.ac.kr}}{\href{https://jeongwhanchoi.me}{Homepage Link}}{\href{https://scholar.google.com/citations?user=3MNElkYAAAAJ&hl=en}{Google Scholar Link}}

%%%% Bio %%%%
\header{Introduction}
\begin{body}
    \vspace{14pt}
 I am a Ph.D. candidate advised by \href{https://scholar.google.com/citations?user=VSuM3gYAAAAJ&hl=en}{Noseong Park} in the Dep. of \emph{Artificial Intelligence} at Yonsei University. I have a broad interest in graph neural networks, recommender systems, spatio-temporal forecasting, and differential equations. Recently, I have been working on developing graph-based deep learning methods inspired by differential equations in natural science.

I was an undergrad at Jeonbuk National University (2016-2020), majoring in \emph{software engineering}. I was privileged to be advised by \href{https://scholar.google.com/citations?user=IN_HTKEAAAAJ&hl=en}{Suntae Kim} and \href{https://scholar.google.com/citations?user=BHue-MMAAAAJ&hl=en}{Duksan Ryu}.
\end{body}

\medskip
%%%% Research Interest %%%%

\header{Research Interest}

\begin{body}
    \vspace{14pt}
    \begin{itemize}
        \item Artificial Intelligence
        \begin{itemize}
            \item Graph Neural Networks
            \item Recommendation (Collaborative Filtering and Sequential Recommendation)
            \item Spatiotemporal Forecasting
            \item Differential Equation-based Deep Learning
        \end{itemize}
        \item Software Engineering
        \begin{itemize}
            \item Software Defect Prediction
            \item AI-based Software Analytics
        \end{itemize}
    \end{itemize}
\end{body}
\medskip

%%%% Work Experience %%%%
\header{Research Experience}
\begin{body}
	\vspace{14pt}
	\textbf{Integrated Ph.D. Student} \hfill \emph{Aug 2020 - Now} \\
	\emph{\href{https://sites.google.com/view/npark}{Big Data Analytics Laboratory (BigDyL)}}, Yonsei University (Advisor: Prof. {\href{https://scholar.google.com/citations?user=VSuM3gYAAAAJ&hl=en}{Noseong Park}})  \\
	% \begin{itemize} \itemsep -0pt  % reduce space between items
 %      \item  
 %  	\end{itemize}
	\textbf{Undergraduate Student Research Assistant} \hfill \emph{Jan 2020 - Aug 2020} \\
	\emph{\href{https://sites.google.com/view/aiselabjbnu}{AI \& SE Lab}}, Jeonbuk National University (Advisor: Prof. {\href{https://scholar.google.com/citations?user=BHue-MMAAAAJ&hl=en}{Duksan Ryu}})  \\
	% \begin{itemize} \itemsep -0pt  % reduce space between items
 %      \item  A Study on Quantitative Software Defect Prediction and Evaluation Techniques Based on Machine Learning
 %  	\end{itemize}
	\textbf{Undergraduate Student Research Assistant} \hfill \emph{Nov 2018 - Nov 2019} \\
	\emph{\href{https://sites.google.com/view/jbnussel/}{Software System and Engineering Laboratory (SSEL)}}, Jeonbuk National University (Advisor: Prof. {\href{https://scholar.google.com/citations?user=IN_HTKEAAAAJ&hl=en}{Suntae Kim}})  \\
	% \begin{itemize} \itemsep -0pt  % reduce space between items
 %      \item  Development of capability assessment evaluation algorithm for personalized self-study with Hanja-Chinese parallel.
 %  	\end{itemize}
\end{body}
\medskip

%%%% Education %%%%
\header{Educational Background}

\begin{body}
	\vspace{14pt}
	\textbf{Integrated Ph.D., Artificial Intelligence} \hfill \emph{Sep 2020 -  Now} \\
	\emph{Yonsei University}, Seoul, Republic of Korea \\
\end{body}

\begin{body}
	\vspace{14pt}
	\textbf{Bachelor, Software Engineering} \hfill \emph{Mar 2016 -  Aug 2020} \\
	\emph{Jeonbuk National University}, Jeonju, Jeollabuk Do, Republic of Korea \\
	\begin{itemize} \itemsep -0pt  % reduce space between items
        \item \textit{magna cum laude} (GPA: 3.98/4.50)
	\end{itemize}
\end{body}

\medskip

%%%% Publication %%%%

\header{Publication}

\begin{body}
	\vspace{14pt}
\presentation{
\textbf{Jeongwhan Choi} and Duksan Ryu, ``QoS-Aware Graph Contrastive Learning for Web Service Recommendation'', In \emph{Proceedings of the 30th Asia-Pacific Software Engineering Conference (APSEC 2023)}, 2023.}{}
\presentation{
\textbf{Jeongwhan Choi} and Noseong Park, ``Graph Neural Rough Differential Equations for Traffic Forecasting,'' \emph{ACM Transactions on Intelligent Systems and Technology (TIST)}, 2023. [\href{https://dl.acm.org/doi/abs/10.1145/3604808}{paper}][\RED{IF=10.489}]}{}
\presentation{
\textbf{Jeongwhan Choi}, Seoyoung Hong, Noseong Park and Sung-Bae Cho, ``GREAD: Graph Reaction-Diffusion Networks,'' \emph{In Proceedings of the 40th International Conference on Machine Learning (ICML)}, 2023. [\href{http://proceedings.mlr.press/v202/choi23a}{paper}][\href{https://github.com/jeongwhanchoi/GREAD}{code}][\RED{Paper Acceptance rate: 27.94\% (1,827/6,538)}]}{}
\presentation{
\textbf{Jeongwhan Choi}, Seoyoung Hong, Noseong Park and Sung-Bae Cho, ``Blurring-Sharpening Process Models for Collaborative Filtering,'' \emph{Proceedings of the 46th ACM Conference on Research and Development in Information Retrieval (SIGIR)}, 2023. [\href{https://arxiv.org/abs/2211.09324}{paper}][\href{https://github.com/jeongwhanchoi/bspm}{code}][\RED{Paper Acceptance rate: 20.1\% (165/822)}]}{}
\presentation{
\textbf{Jeongwhan Choi} and Duksan Ryu, ``Graph Convolution-based Collaborative Filtering for Web Service QoS Ranking'', In \emph{Proceedings of the 25th Korea Conference on Software Engineering (KCSE 2023)}, 2023, pp. 58-67.}{}
\presentation{
Hwangyong Choi, \textbf{Jeongwhan Choi}, Jeehyun Hwang, Kookjin Lee, Dongeun Lee and Noseong Park, ``Climate Modeling with Neural Advection-Diffusion Equation,'' \emph{Knowledge and Information Systems}, Jan. 2023. [\href{https://doi.org/10.1007/s10115-023-01829-2}{paper}] [\RED{IF=3.205(2021) Five year impact factor}]}{}

\presentation{
Jaehoon Lee, Chan Kim, Gyumin Lee, Haksoo Lim, \textbf{Jeongwhan Choi}, Kookjin Lee, Dongeun Lee, Sanghyun Hong and Noseong Park, ``Time Series Forecasting with Hypernetworks Generating Parameters in Advance,'' \emph{arXiv preprint arXiv: Arxiv-2211.12034}, 2022. [\href{https://arxiv.org/abs/2211.12034}{paper}]}{}
\presentation{Seoyoung Hong, Heejoo Shin, \textbf{Jeongwhan Choi}, and Noseong Park, ``Prediction-based One-shot Dynamic Parking Pricing,'' In \emph{Proceedings of the 31st ACM International Conference on Information and Knowledge Management (CIKM)}, 2022.[\href{https://arxiv.org/abs/2208.14231}{paper}][\href{https://github.com/jeongwhanchoi/one-shot-optimization}{code}]}{}
\presentation{\textbf{Jeongwhan Choi}, Hwangyong Choi, Jeehyun Hwang and Noseong Park, ``Graph Neural Controlled Differential Equations for Traffic Forecasting,'' In \emph{AAAI}, 2022. [\href{https://ojs.aaai.org/index.php/AAAI/article/download/20587/20346}{paper}][\href{https://github.com/jeongwhanchoi/STG-NCDE}{code}][\RED{Regular Paper Acceptance rate: 14.2\% (1,161/8,198)}] [\RED{Overall Acceptance rate: 15.2\% (1,370/9,020)] }} {}
\presentation{Taeyong Kong, Taeri Kim, Jinsung Jeon, \textbf{Jeongwhan Choi}, Yeon-Chang Lee, Noseong Park and Sang-Wook Kim, ``Linear, or Non-Linear, That is the Question!,'' In Proceedings of the 15th ACM International Web Search and Data Mining Conference (WSDM), 2022.  [\href{https://arxiv.org/abs/2111.07265}{paper}][\href{https://github.com/jeongwhanchoi/HMLET}{code}] [\RED{Regular Paper Acceptance rate: 15.8\% (80/505)}] [\RED{Overall Acceptance Rate: 18\%  (315/1,765)} ]}{}
\presentation{\textbf{Jeongwhan Choi} and Duksan Ryu, ``Self-Supervised Learning Using Feature Subsets of Software Defect Data'',  In \emph{Proceedings of the Korea Software Congress (KSC)}, Dec. 2021, pp.203-205.}{}
\presentation{Jeehyun Hwang, \textbf{Jeongwhan Choi}, Hwangyong Choi, Kookjin Lee, Dongeun Lee and Noseong Park, ``Climate Modeling with Neural Diffusion Equations'', In \emph{Proceedings of the 21st IEEE International Conference on Data Mining (ICDM)}, 2021.  [\href{https://arxiv.org/abs/2111.06011}{paper}] [\href{https://github.com/jeongwhanchoi/Neural-Diffusion-Equation}{code}] [\RED{Regular paper acceptance rate: 9.9\%  (98/990)}] [\RED{Overall Acceptance Rate: 20\%  (198/990)}]}{}
\presentation{\textbf{Jeongwhan Choi} and Duksan Ryu, ``Bayesian Optimization Framework for Improved Cross-Version Defect Prediction'', \emph{KIPS Transactions on Software and Data Engineering (KTSDE)}, Vol. 10, No. 9, pp. 339-348, Sep. 2021.}{}
\presentation{\textbf{Jeongwhan Choi}, Jinsung Jeon, and Noseong Park, ``LT-OCF: Learnable-Time ODE-based Collaborative Filtering'', In \emph{Proceedings of the 30th ACM International Conference on Information and Knowledge Management (CIKM)}, 2021. [\href{https://arxiv.org/pdf/2108.06208.pdf}{paper}] [\href{https://github.com/jeongwhanchoi/LT-OCF}{code}]  [\RED{Regular paper acceptance rate: 21.7\% (271/1,251)}]  [\RED{Overall Acceptance rate: 22\% (1,101/4,989)]} }{}
\presentation{\textbf{Jeongwhan Choi} and Duksan Ryu, ``Bayesian Optimization Framework for Cross-Version Defect Prediction'', In \emph{Proceedings of the 23rd Korea Conference on Software Engineering (KCSE 2021)}, 2021, pp. 63-72.  [\RED{Best Paper}]}{}
\presentation{\textbf{Jeongwhan Choi}, Jiwon Choi, Duksan Ryu and Suntae Kim, ``Improved Prediction for Configuration Bug Report Using Text Mining and Dimensionality Reduction,'' \emph{Journal of KIISE}, 2021, Vol. 48, No. 1, pp. 35-42.}{}
\presentation{\textbf{Jeongwhan Choi} and Duksan Ryu, ``A Study on the Applicability of Transfer Learning Techniques for Cross-Project Defect Regression,'' In \emph{Proceedings of the Korea Software Congress (KSC)}, 2020, pp. 150 - 152.}{}
\presentation{\textbf{Jeongwhan Choi}, Duksan Ryu, and Suntae Kim, ``Comparative Study of Transfer Learning Models for Cross-Project Automotive Software Defect Prediction,'' In \emph{Proceedings of the Korea Computer Congress (KCC)}, 2020, pp. 257–259.}{}
\presentation{\textbf{Jeongwhan Choi}, Jiwon Choi, Duksan Ryu, and Suntae Kim, ``Prediction for Configuration Bug Report Using Text Mining,'' In \emph{Proceedings of the 22nd Korea Conference on Software Engineering (KCSE 2020)}, 2020, pp. 350–357.}{}
\presentation{\textbf{Jeongwhan Choi}, Jiwoo Noh, and Suntae Kim, ``Prediction Techniques for Difficulty Level of Hanja Using Multiple Linear Regression,'' \emph{J. Inst. Internet, Broadcast. Commun.}, vol. 19, no. 6, 2019.}{}
\presentation{Seounghan Song, \textbf{Jeongwhan Choi}, Mingu Kang, and Cheoljung Yoo, ``A Software Module That Analyzes the Relationship Between Headline and Content of the Web Article: CHIMERA,'' in \emph{Proceedings of the 2019 KIIT DCS Summer Conference}, 2019, vol. 14, pp. 437–440.}{}
\presentation{\textbf{Jeongwhan Choi}, ``Iceberg-Ship Classification in SAR Images Using Convolutional Neural Network with Transfer Learning,'' \emph{J. Internet Comput. Serv.}, vol. 19, no. 4, pp. 35–44, 2018.}{}

\end{body}

\medskip
% \pagebreak
%%%% Honors & Awards %%%%

\header{Awards \& Scholarships}

\begin{body}
	\vspace{14pt}

\award{Innovation Award, Yonsei University (Best paper in Dept. of Artificial Intelligence) \href{https://www.yonsei.ac.kr/ocx/news.jsp?mode=view&ar_seq=20220708141917269049&sr_volume=632&list_mode=list&sr_site=S&pager.offset=0&sr_cates=29}{[link]}}{Jul 2022}
\award{Best Paper Awards in the 23rd Korea Conference on Software Engineering (KCSE 2021)}{Feb 2021}
\award{Best Paper Awards in Dep. of Software Engineering}{Dec 2019}
\award{Best Paper Awards, Korean Institute of Information Technology}{Jun 2019}
\award{The National Scholarship for Science and Engineering, KOSAF(Korea Student Aid Foundation)}{2018-2019}
    \begin{itemize} \itemsep -0pt  % reduce space between items
      \item This scholarship supports undergraduates with strong academic performance in science and engineering, with the purpose of developing future leaders in those fields.
  	\end{itemize}
\award{Academic Excellent Scholarship}{2016-2019}
\end{body}

\medskip

%%%% Talks %%%%
\header{Talks}

\begin{body}
	\vspace{14pt}
\award{Talk on 1st Seminar held by Graph User Group (GUG)}{Jun 2023}
\award{Invited talk on Top-conference session, Korea Computer Congress (KCC 2023) [\href{https://www.dropbox.com/s/34h6pmr7ftdiuzr/BSPM-KCC23.pptx?dl=0}{slides}]}{Jun 2023}
\award{Talk on 2023 KSIAM AI Winter School, held by Korean Society for Industrial and Applied Mathematics (KSIAM)[\href{https://www.dropbox.com/s/p4sd5h40hcuxcob/KSIAM23-Tutorial-ODE-RecSys.pdf?dl=0}{slides}][\href{https://ksiam.org/Conference/ConferenceView.asp?AC=3&CODE=CD20230101&CpPage=\#CONF}{website}]}{Feb 2023}
\award{Talk on 1st Workshop on AI held by Yonsei Univ. [\href{https://www.dropbox.com/s/9au5xx13qa2l529/AAAI22_workshop.pdf?dl=0}{slides}][\href{https://www.dropbox.com/s/pibzd51d76zy907/AAAI22-Yonsei_AI_Workshop.pdf?dl=0}{poster}]}{Oct 2022}
\award{Poster presentation for AIGS Symposium 2022 held at the COEX Grand Ballroom [\href{https://www.dropbox.com/s/gfjsizak9s4cn9o/AAAI22-AIGS.pdf?dl=0}{poster}]}{Aug 2022}
\award{Invited talk on Top-conference session,  Korea Computer Congress (KCC 2022) [\href{https://www.dropbox.com/s/22d9d92ns8uv9qw/AAAI22_KCC22.pdf?dl=0}{slides}]}{Jul 2022}
\award{Tutorial on Korea Artificial Intelligence Association (KAIA)}{Nov 2021}
	\begin{itemize} \itemsep -0pt  % reduce space between items
        \item Topic: ``Graph-based Collaborative Filtering and Neural ODEs''
		\item This talk is part of a tutorial called ``Deep Learning Inspired by Differential Equation'' [\href{https://www.dropbox.com/s/1xn8xhd6llmhblz/%5BKAIA2021%5DTutorial-LT-OCF.pdf?dl=0}{slides}].
  	\end{itemize}
\end{body}

\medskip

%%%% Service %%%%
\header{Service}

\begin{body}
	\vspace{14pt}
    \begin{itemize}
        \item Reviewer in SDM 2024
        \item Reviewer in AAAI 2023
        \item Reviewer in Applied Artificial Intelligence 
        \item Reviewer in KDD 2023
        \item Reviewer in Learning on Graph Conference (LoG) 2022, 2023
        \item Reviewer in IEEE Transactions on Intelligent Transportation Systems
        \item Reviewer in ICDM 2021, 2022
    \end{itemize}
\end{body}

\medskip

\pagebreak
%%%% Patent %%%%
\header{Patent and S/W Program}

\begin{body}
	\vspace{14pt}
	
\presentation{[Issued Patent] Apparatus and Method for Processing Spatiotemporal Data Based on Graph Neural Controlled Differential Equations, Noseong Park, \textbf{Jeongwhan Choi}, Jeehyun Hwang, Hwangyong Choi, U.S.A. Patent(Issued Number: 18/085,109). 2022.12.20}{Dec 2022}	
\presentation{[Issued Patent] Apparatus and Method for Processing Spatiotemporal Data Based on Graph Neural Controlled Differential Equations, Noseong Park, \textbf{Jeongwhan Choi}, Jeehyun Hwang, Hwangyong Choi, Domestic Patent(Issued Number: 10-2022-0151819). 2022.11.14}{Nov 2022}	
\presentation{[S/W] LT-OCF: Learnable-Time ODE-based Collaborative Filtering, Korea Copyright Commission, C-2021-052779, 2021.12.}{Dec 2021}	
\presentation{[Issued Patent] Apparatus and Method for Collaborative Filtering Based on Learnable-Time Ordinary Differential Equation, Noseong Park, \textbf{Jeongwhan Choi}, Jinsung Jeon, Japan Patent(Issued Number: 2021-215162). 2021.12.28}{Dec 2021}	
\presentation{[Issued Patent] Apparatus and Method for Collaborative Filtering Based on Learnable-Time Ordinary Differential Equation, Noseong Park, \textbf{Jeongwhan Choi}, Jinsung Jeon, U.S.A. Patent(Issued Number: 	17/563,726). 2021.12.28}{Dec 2021}	
\presentation{[Issued Patent] Apparatus and Method for Collaborative Filtering Based on Learnable-Time Ordinary Differential Equation, Noseong Park, \textbf{Jeongwhan Choi}, Jinsung Jeon, Domestic Patent(Issued Number: 10-2021-0177928). 2021.12.13}{Dec 2021}	
\presentation{[Granted Patent] Apparatus and Method for Measuring Difficulty Level of Chinese Character Using Regression Analysis, Suntae Kim, \textbf{Jeongwhan Choi}, Jiwoo Noh, Domestic Patent(Application Number:10-2019-0141339 ). 2019.11. \href{https://doi.org/10.8080/1020190141339}{[link]}}{Nov 2019}	
  
\end{body}
    
\medskip

%%%% Certifications %%%%

\header{Certifications}

\begin{body}
	\vspace{14pt}

  %% UMLC Jeju  %%
  	\labtitle{University Machine Learning Camp in Jeju}{Jeju University}{Aug 2020}
  \labdescription {
  	\begin{itemize} \itemsep -0pt  % reduce space between items
      \item \href{https://drive.google.com/file/d/1lV5w4cdCMVgFnBGu2sHeTigXFEZCX3Sr/view}{See credential}
  	\end{itemize}
  }
  %% IBM Blockchain Foundation for Developers  %%
  	\labtitle{IBM Blockchain Foundation for Developers }{Coursera}{Feb 2018 - Present}
  \labdescription {
  	\begin{itemize} \itemsep -0pt  % reduce space between items
      \item License 5MMQUBFWE2K3 (\href{https://www.credly.com/badges/a9f84e03-9bc0-462e-8dd2-66ed9c7878c0/linked_in_profile}{See credential})
  	\end{itemize}
  }
  
%% Machine Learning Engineer Nanodegree %%
  	\labtitle{Machine Learning Engineer Nanodegree}{Udacity}{Jan 2018 - Present}
  \labdescription {
  	\begin{itemize} \itemsep -0pt  % reduce space between items
        \item \href{https://graduation.udacity.com/confirm/AN2EMAQD}{See credential}
  	\end{itemize}
  }

%% Machine Learning%%
  	\labtitle{Machine Learning}{Coursera}{July 2017 - Present}
  \labdescription {
  	\begin{itemize} \itemsep -0pt  % reduce space between items
      \item License EEYYGQPCFLN7 (\href{https://www.coursera.org/account/accomplishments/verify/EEYYGQPCFLN7}{See credential})
  	\end{itemize}
  }
  


\end{body}

\medskip

%%%% Skills %%%%
\header{Skills}

\begin{body}
	\vspace{14pt}
%% Tools & Technologies %%
  	\labtitle{Tools \& Technologies}{}{}
  \labdescription {
  	\begin{itemize} \itemsep -0pt  % reduce space between items
      \item PyTorch, TensorFlow, Python
      \item Java, C/C++, R, LaTeX, VBA, Unified Modeling Language
      \item Android, Matlab, Git, RSA
      \item MySQL, Tomcat, JSP, HTML, Javascript, JUnit
  	\end{itemize}
  }
  %% Industry Knowledge. %%
  	\labtitle{Industry Knowledge}{}{}
  \labdescription {
  	\begin{itemize} \itemsep -0pt  % reduce space between items
      \item Artificial Intelligence, Graph Neural Networks, Neural Ordinary Differential Equations (Neural ODEs), Recommender Systems, Time-series Forecasting, Spatio-temporal Forecasting, Software Defect Prediction
      \item Software Engineering, Object Oriented Programming, Design Pattern, Compiler, Software Testing(Static Analysis)
      \item Text Mining, Image Processing for SAR Image
      \item ARM Cortex-M3, ESP-8266
  	\end{itemize}
  }

\end{body}

\medskip

%%%% Projects %%%%

\header{Projects (From 2017 to 2019)}

\begin{body}
	\vspace{14pt}
\project{Prediction for Configuration Bug Report Using Text Mining}{Nov 2019 - Dec 2019}
\begin{itemize} \itemsep -0pt  % reduce space between items
      \item  The purpose of this project is to predict the configuration bug reports using machine learning techniques and NLP.
  	\end{itemize}
\project{Development of capability assessment evaluation algorithm for personalized self-study with Hanja-Chinese parallel}{Dec 2018 - Expected Nov 2019}
\begin{itemize} \itemsep -0pt  % reduce space between items
      \item  The purpose of this project is to solve the problems of existing Hanja character difficulty selection method.
      \item It includes the technique for measuring the difficulty of Hanja characters using artificial intelligence.
      \item It also covers personalized learning induction technique using a clustering model.
  	\end{itemize}
\project{Stock Price Prediction Model Based LSTM to Maximize Return on Investment}{Oct 2018 - Dec 2018}
\begin{itemize} \itemsep -0pt  % reduce space between items
      \item  The purpose of this project is to predict the long-term stock flow based on the AI prediction model and to derive meaningful ROI.
      \item This project has a paper which is not submitted.
      \item \href{https://drive.google.com/file/d/1Fp3WmoBpHpYU13fNWS1vbQSzmBCR3MIu/view}{See project}
  	\end{itemize}
\project{Advanced Lane Finding Project}{Jul 2018}
\begin{itemize} \itemsep -0pt  % reduce space between items
      \item  The goal is to find the lane line using advanced image processing techniques.
      \item \href{https://github.com/jeongwhanchoi/CarND-Advanced-Lane-Lines}{See project}
  	\end{itemize}
\project{Vehicle Detection}{Jul 2018}
\begin{itemize} \itemsep -0pt  % reduce space between items
      \item  The software pipeline using computer vision algorithms to detect vehicles in videos.
      \item \href{https://github.com/jeongwhanchoi/CarND-Vehicle-Detection}{See project}
  	\end{itemize}
\project{Clone Driving  Behavior}{Jun 2018}
\begin{itemize} \itemsep -0pt  % reduce space between items
      \item  The goal is to clone driving behavior via the CNN model.
      \item \href{https://github.com/jeongwhanchoi/CarND-Behavioral-Cloning}{See project}
  	\end{itemize}
\project{Recipe Assistant App}{Apr 2018 - Jun 2018}
\begin{itemize} \itemsep -0pt  % reduce space between items
      \item  The recipe assistant app which helps people to cook an easy way. This app was implemented by Android and improved by several design patterns.
      \item \href{https://github.com/jeongwhanchoi/recipe-assistant-app}{See project}
  	\end{itemize}
\project{Tic-Tac-Toe game for LPC 1768}{Jun 2018}
\begin{itemize} \itemsep -0pt  % reduce space between items
      \item  This project is the Tic-Tac-Toe Game using ARM Cortex-M3(LPC 1768)
      \item \href{https://github.com/jeongwhanchoi/tic-tac-toe-lpc1768}{See project}
  	\end{itemize}
\project{Lane Finding Project}{Mar 2018}
\begin{itemize} \itemsep -0pt  % reduce space between items
      \item  The goal is to find the lane lines on the road.
      \item \href{https://github.com/jeongwhanchoi/CarND-LaneLines}{See project}
  	\end{itemize}
\project{Iceberg Classifier}{Jan 2018}
\begin{itemize} \itemsep -0pt  % reduce space between items
      \item  The goal is to create an image classification model that finds icebergs among SAR images collected by satellites. This project has a paper published in JICS.
      \item \href{https://github.com/jeongwhanchoi/MLND-Capstone-Project}{See project}
  	\end{itemize}
\project{Helicopter Battle Game}{Apr 2017 - Jul 2017}
\begin{itemize} \itemsep -0pt  % reduce space between items
      \item  This Java project includes the software engineering knowledges such as objective-oriented design, design patterns and so on.
      \item \href{https://github.com/jeongwhanchoi/helicopter_battle}{See project}
  	\end{itemize}
\project{Smart Mailbox}{Sep 2017 - Dec 2017}
\begin{itemize} \itemsep -0pt  % reduce space between items
      \item  The smart mailbox project played a role in helping people who forget offline mails through notifications to mobile phones.
      \item \href{https://github.com/jeongwhanchoi/Smart-Mailbox}{See project}
  	\end{itemize}	
\end{body}


\medskip


\end{document}


