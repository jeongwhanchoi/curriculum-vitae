\documentclass[10pt]{article}
\usepackage{fullpage}
\usepackage{amsmath}
\usepackage{amssymb}
\usepackage[usenames,dvipsnames]{xcolor}
\usepackage[]{hyperref}
\usepackage{kotex}
\hypersetup{
    colorlinks=true,
    citecolor=black,
    filecolor=black,
    linkcolor=NavyBlue,
    urlcolor=NavyBlue,
    breaklinks=true
}
\urlstyle{same}

\leftmargin=0.25in
\oddsidemargin=0.25in
\textwidth=6.0in
\topmargin=-0.25in
\textheight=9.25in

\raggedright

\pagenumbering{gobble} % suppresses page numbering. To stop suppression, change to 'arabic'.

\def\bull{\vrule height 0.8ex width .7ex depth -.1ex }


% DEFINITIONS FOR CV - defining some features to simplify template

\newenvironment{changemargin}[2]{
  \begin{list}{}{
    \setlength{\topsep}{0pt}
    \setlength{\leftmargin}{#1}
    \setlength{\rightmargin}{#2}
    \setlength{\listparindent}{\parindent}
    \setlength{\itemindent}{\parindent}
    \setlength{\parsep}{\parskip}
  }
  \item[]}{\end{list}
}

\newcommand{\lineover}{
	\begin{changemargin}{-0.05in}{-0.05in}
		\vspace*{-8pt}
		\hrulefill \\
		\vspace*{-2pt}
	\end{changemargin}
}

\newcommand{\header}[1]{
	\begin{changemargin}{-0.5in}{-0.5in}
		\scshape{#1}\\
  	\lineover
	\end{changemargin}
}

\newcommand{\contact}[5]{
	\begin{changemargin}{-0.5in}{-0.5in}
		\begin{center}
			{\Large \scshape {#1}}\\
      {#2} \\  {#3} \\ {#4} \\ {#5}
		\end{center}
	\end{changemargin}
}

\newcommand{\labdescription}[1]{
	\begin{changemargin}{0.15in}{0.15in}
    \smallskip
		{#1}
    \medskip
	\end{changemargin}
}

\newcommand{\labtitle}[3]{
	\textbf{#1}, \emph{#2} \hfill \emph{#3}\\
}

\newcommand{\presentation}[2]{
	{#1} \hfill \emph{#2}\\ \bigskip
}

\newcommand{\award}[2]{
	{#1} \hfill \emph{#2}\\ \medskip
}

\newcommand{\society}[2]{
	{#1} \hfill \emph{#2}\\ \medskip
}

\newcommand{\project}[2]{
	{#1} \hfill \emph{#2}\\ \medskip
}

\newenvironment{body} {
	\vspace*{-16pt}
	\begin{changemargin}{-0.25in}{-0.5in}
  }
	{\end{changemargin}
}

% END CV DEFINITIONS

\begin{document}

%%%% Name %%%%

\contact{Jeongwhan Choi}{\href{mailto:email@address.com}{jeongwhan.choi@jbnu.ac.kr}}{+82 1036961205}{\href{https://jeongwhanchoi.github.io}{Personal Page Link}}

%%%% Education %%%%

\header{Education}

\begin{body}
	\vspace{14pt}
	\textbf{Bachelor, Software Engineering} \hfill \emph{Mar 2016 - Expected Aug 2020} \\
	\emph{Jeonbuk National University}, Jeonju Si, Jeollabuk Do, Republic of Korea \\
	\begin{itemize} \itemsep -0pt  % reduce space between items
	\item GPA: 3.98/4.50
	\end{itemize}

\end{body}

\medskip

%%%% Work Experience %%%%

\header{Work Experience}

\begin{body}
	\vspace{14pt}
	\textbf{Undergraduate Student Research Assistant} \hfill \emph{Jan 2020 - Expected June 2020} \\
	\emph{AI \& SE Lab}, Jeonbuk National University (Advisor: Prof. {\href{https://www.linkedin.com/in/duksanryu}{Duksan Ryu}})  \\
	\begin{itemize} \itemsep -0pt  % reduce space between items
      \item  A Study on Quantitative Software Defect Prediction and Evaluation Techniques Based on Machine Learning
  	\end{itemize}
	
	\textbf{Undergraduate Student Research Assistant} \hfill \emph{Nov 2018 - Nov 2019} \\
	\emph{SSEL(Software System and Engineering Laboratory)}, Jeonbuk National University (Advisor: Prof. {\href{https://sites.google.com/site/jipsin08/}{Suntae Kim}})  \\
	\begin{itemize} \itemsep -0pt  % reduce space between items
      \item  Development of capability assessment evaluation algorithm for personalized self-study with Hanja-Chinese parallel.
  	\end{itemize}


\end{body}

\medskip

%%%% Skills %%%%

\header{Skills}

\begin{body}
	\vspace{14pt}

%% Tools & Technologies %%
  	\labtitle{Tools \& Technologies}{}{}
  \labdescription {
  	\begin{itemize} \itemsep -0pt  % reduce space between items
      \item Java, Python, C/C++, R, LaTeX, VBA, Unified Modeling Language
      \item Matlab, Android, Jupyter Notebook, Git, RSA
      \item MySQL, Tomcat, JSP, HTML, Javascript, JUnit
  	\end{itemize}
  }
  
  %% Industry Knowledge. %%
  	\labtitle{Industry Knowledge}{}{}
  \labdescription {
  	\begin{itemize} \itemsep -0pt  % reduce space between items
      \item Software Engineering, Object Oriented Programming, Design Pattern, Compiler, Software Testing(Static Analysis), Programming Language
      \item Machine Learning, Deep Learning, Data Science
      \item Computer Vision, Natural Language Processing
      \item ARM Cortex-M3, ESP-8266
  	\end{itemize}
  }
  
   %% English. %%
  	\labtitle{English}{}{}
  \labdescription {
  	\begin{itemize} \itemsep -0pt  % reduce space between items
      \item TOEIC 860
  	\end{itemize}
  }

\end{body}

\medskip

%%%% Publication %%%%

\header{Publication}

\begin{body}
	\vspace{14pt}
	
\presentation{Prediction for Configuration Bug Report Using Text Mining\\ \emph{Proceedings of the 22nd Korea Conference on Software Engineering (KCSE 2020)}, 22 (1), 376-383.}{Feb 2020}

\presentation{Prediction Techniques for Difficulty Level of Hanja Using Multiple Linear Regression\\ \emph{The Journal of the Institute of Internet, Broadcasting and Communication}, 22, 219-225.}{Dec 2019}

\presentation{A Software Module That Analyzes the Relationship Between Headline and Content of the Web Article: CHIMERA.\\ \emph{The Proceedings of the 2019 KIIT DCS Summer Conference}, 14 (1), 437-440.}{Jun 2019}

\presentation{Iceberg-Ship Classification in SAR Images Using Convolutional Neural Network with Transfer Learning\\ \emph{JICS(Journal of Internet Computing and Services)}, 19 (4), 35-44.}{Sep 2018}


\end{body}

\medskip

\pagebreak

%%%% Patent %%%%

\header{Patent}

\begin{body}
	\vspace{14pt}
	
\presentation{회귀 분석을 이용한 한자 난이도 측정 장치 및 방법 , 김순태, 최정환, 노지우\\ \emph{국내특허(출원번호:10-2019-0141339)}}{Nov 2019}	
  
\end{body}

\medskip


%%%% Honors & Awards %%%%

\header{Honors \& Awards}

\begin{body}
	\vspace{14pt}

\award{The National Scholarship for Science and Engineering}{2018-2019}
\begin{itemize} \itemsep -0pt  % reduce space between items
      \item  KOSAF(Korea Student Aid Foundation)
      \item This is the type of merit-based aid for four semesters
      \item This scholarship supports undergraduates with strong academic performance in science and engineering, with the purpose of developing future leaders in those fields.
  	\end{itemize}
	
	\award{Academic Excellent Scholarship}{2016-2019}
\begin{itemize} \itemsep -0pt  % reduce space between items
      \item  Jeonbuk National University
      \item Receive a scholarship for the best grade during the four semesters.
  	\end{itemize}
	
	
	\award{대학생 논문 경진대회 우수 논문}{Jun 2019}
	\begin{itemize} \itemsep -0pt  % reduce space between items
		\item  KIIT(한국정보기술학회)
		\item A Software Module That Analyzes the Relationship Between Headline and Content of the Web Article: CHIMERA
  	\end{itemize}
	
	\award{Software Engineering Day 우수상}{Dec 2019}
	\begin{itemize} \itemsep -0pt  % reduce space between items
		\item  전북대학교 소프트웨어공학과 주최
		\item A Software Module That Analyzes the Relationship Between Headline and Content of the Web Article: CHIMERA
  	\end{itemize}

\end{body}


\medskip


%%%% Certifications %%%%

\header{Certifications}

\begin{body}
	\vspace{14pt}

%% Machine Learning%%
  	\labtitle{Machine Learning}{Coursera}{July 2017 - Present}
  \labdescription {
  	\begin{itemize} \itemsep -0pt  % reduce space between items
      \item License EEYYGQPCFLN7
  	\end{itemize}
  }
  
  %% Machine Learning Engineer Nanodegree %%
  	\labtitle{Machine Learning Engineer Nanodegree}{Udacity}{Jan 2018 - Present}
  \labdescription {
  	\begin{itemize} \itemsep -0pt  % reduce space between items
  	\end{itemize}
  }
  
  %% IBM Blockchain Foundation for Developers  %%
  	\labtitle{IBM Blockchain Foundation for Developers }{Coursera}{Feb 2018 - Present}
  \labdescription {
  	\begin{itemize} \itemsep -0pt  % reduce space between items
      \item License 5MMQUBFWE2K3
  	\end{itemize}
  }

\end{body}

\medskip


%%%% Projects %%%%

\header{Projects}

\begin{body}
	\vspace{14pt}

\project{Smart Mailbox}{Sep 2017 - Dec 2017}
\begin{itemize} \itemsep -0pt  % reduce space between items
      \item  This project is the smart mailbox notifies a user when a new mail arrives at the mailbox.
      \item \href{https://github.com/jeongwhanchoi/Smart-Mailbox}{See project}
  	\end{itemize}
	
\project{Helicopter Battle Game}{Apr 2017 - Jul 2017}
\begin{itemize} \itemsep -0pt  % reduce space between items
      \item  This project is the game improvement project in Java.
      \item \href{https://github.com/jeongwhanchoi/helicopter_battle}{See project}
  	\end{itemize}

\project{Iceberg Classifier}{Jan 2018}
\begin{itemize} \itemsep -0pt  % reduce space between items
      \item  The goal is to create an image classification model that finds icebergs among SAR images collected by satellites.
      \item This project has a paper published in JICS.
      \item \href{https://github.com/jeongwhanchoi/MLND-Capstone-Project}{See project}
  	\end{itemize}

\project{Lane Finding Project}{Mar 2018}
\begin{itemize} \itemsep -0pt  % reduce space between items
      \item  The goal is to find the lane lines on the road.
      \item \href{https://github.com/jeongwhanchoi/CarND-LaneLines}{See project}
  	\end{itemize}

\project{Tic-Tac-Toe game for LPC 1768}{Jun 2018}
\begin{itemize} \itemsep -0pt  % reduce space between items
      \item  This project is the Tic-Tac-Toe Game using ARM Cortex-M3(LPC 1768)
      \item \href{https://github.com/jeongwhanchoi/tic-tac-toe-lpc1768}{See project}
  	\end{itemize}
	
\project{Recipe Assistant App}{Apr 2018 - Jun 2018}
\begin{itemize} \itemsep -0pt  % reduce space between items
      \item  This project is the recipe assistant app which helps people to cook an easy way.
      \item \href{https://github.com/jeongwhanchoi/recipe-assistant-app}{See project}
  	\end{itemize}
	
		
\project{Clone Driving  Behavior}{Jun 2018}
\begin{itemize} \itemsep -0pt  % reduce space between items
      \item  The goal is to clone driving behavior via the CNN model.
      \item \href{https://github.com/jeongwhanchoi/CarND-Behavioral-Cloning}{See project}
  	\end{itemize}
	
\project{Vehicle Detection}{Jul 2018}
\begin{itemize} \itemsep -0pt  % reduce space between items
      \item  The Software Pipeline to Detect Vehicles in a Video.
      \item \href{https://github.com/jeongwhanchoi/CarND-Vehicle-Detection}{See project}
  	\end{itemize}
	
\project{Advanced Lane Finding Project}{Jul 2018}
\begin{itemize} \itemsep -0pt  % reduce space between items
      \item  The goal is to find the lane line using advanced techniques.
      \item \href{https://github.com/jeongwhanchoi/CarND-Advanced-Lane-Lines}{See project}
  	\end{itemize}
	
\project{Stock Price Prediction Model Based LSTM to Maximize Return on Investment}{Oct 2018 - Dec 2018}
\begin{itemize} \itemsep -0pt  % reduce space between items
      \item  The purpose of this project is to predict the long-term stock flow based on the AI prediction model and to derive meaningful ROI.
      \item This project has a paper which is not submitted.
      \item \href{https://drive.google.com/file/d/1Fp3WmoBpHpYU13fNWS1vbQSzmBCR3MIu/view}{See project}
  	\end{itemize}
	
\project{Development of capability assessment evaluation algorithm for personalized self-study with Hanja-Chinese parallel(한자-중국어 병행 맞춤형 자기 학습을 위한 학습 역량 진단 평가 알고리즘 개발)}{Dec 2018 - Expected Nov 2019}
\begin{itemize} \itemsep -0pt  % reduce space between items
      \item  The purpose of this project is to solve the problems of existing Hanja character difficulty selection method.
      \item It includes the technique for measuring the difficulty of Hanja characters using artificial intelligence.
      \item It also covers personalized learning induction technique using a clustering model.
      \item A patent was derived from this project. "회귀 분석을 이용한 한자 난이도 측정 장치 및 방법
(APPARATUS AND METHOD FOR MEASURING DIFFICULTY LEVEL OF CHINESE CHARACTER USING REGRESSION ANALYSIS)"
  	\end{itemize}
	
\project{Prediction for Configuration Bug Report Using Text Mining}{Nov 2019 - Dec 2019}
\begin{itemize} \itemsep -0pt  % reduce space between items
      \item  The purpose of this project is to predict the configuration bug reports using machine learning techniques and NLP.
  	\end{itemize}
	
\end{body}


\medskip


\end{document}


